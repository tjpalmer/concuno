\documentclass[12pt,twoside]{article}

\usepackage{amsmath}
\usepackage{amstext}
\usepackage[letterpaper,hmargin={1in,1in},vmargin={1in,1in}]{geometry}
\usepackage[version=3]{mhchem}

\newcommand{\term}[1]{\emph{#1}}
\newcommand{\sspre}[3]{\ce{^{\mathit{#2}}_{\mathit{#3}}}\hspace{-1pt}#1}

\title{Concuno Description}
\author{Thomas J. Palmer}

\begin{document}
\maketitle

The name comes from the Spanish \term{concu\~no}, meaning the spouse of a
sibling-in-law. In English, there is no word for this precise relationship, and
hence the name of this software. Its goal is to discover interesting
relationships automatically, although with a focus on physical relationships
rather than symbolic ones. I suspect \term{concu\~no} is best defined in
terms of other simpler already-symbolic relations, such as what I did above.
It doesn't fit perfectly, but it was still nice enough.

Also, I changed the \term{\~n} to an \term{n} on purpose. It's hard for me and
some others to bother going outside 7-bit ASCII characters. And now I'm not
officially using a real word (that I know of) outright, even if it matches well
enough for web searches.

Concuno has many concepts founded in the Spatiotemporal Multidimensional
Relational Framework (\term{SMRF}) of Bodenhamer et al.\ and includes a partial,
modified implementation of SMRF.


\section{Metric on two leaves}

\begin{align*}
  Y^+ &= \text{\# of positive yes bags (fully \emph{in}side the boundary)} \\
  N^+ &= \text{\# of positive no bags (fully \emph{out}side the boundary)} \\
  M^+ &=
    \text{\# of positive maybe bags (both inside and outside the boundary)} \\
  Y^-, N^-, M^- &=
    \text{\# of \term{negative} yes, no, and maybe bags, respectively} \\
  Y &= Y^+ + Y^- \\
  N &= N^+ + N^- \\
  M &= M^+ + M^-
\end{align*}

\begin{multline}
  \left(\frac{Y^+ + M^+}{Y + M}\right)^{Y^+ + M^+}
  \left(1 - \frac{Y^+ + M^+}{Y + M}\right)^{Y^- + M^-}
  \left(\frac{N^+}{N}\right)^{N^+}
  \left(1 - \frac{N^+}{N}\right)^{N^-}
  \stackrel{?}{>} \\
  \left(\frac{Y^+}{Y}\right)^{Y^+}
  \left(1 - \frac{Y^+}{Y}\right)^{Y^-}
  \left(\frac{N^+ + M^+}{N + M}\right)^{N^+ + M^+}
  \left(1 - \frac{N^+ + M^+}{N + M}\right)^{N^- + M^-}
\end{multline}

\begin{align*}
  Y^+ &= \text{\# of positive yes bags (fully \emph{in}side the boundary)} \\
  N^+ &= \text{\# of positive no bags (fully \emph{out}side the boundary)} \\
  M^+ &=
    \text{\# of positive maybe bags (both inside and outside the boundary)} \\
  Y^-, N^-, M^- &=
    \text{\# of \term{negative} yes, no, and maybe bags, respectively} \\
  Y &= Y^+ + Y^- \\
  N &= N^+ + N^- \\
  M &= M^+ + M^- \\
  \sspre{P}{Y}{YM} &= (Y^+ + M^+) / (Y + M) \\
  \sspre{P}{N}{YM} &= N^+ / N \\
  \sspre{P}{Y}{NM} &= Y^+ / Y \\
  \sspre{P}{N}{NM} &= (N^+ + M^+) / (N + M)
\end{align*}

\begin{multline}
  \left(\sspre{P}{Y}{YM}\right)^{Y^+ + M^+}
  \left(1 - \sspre{P}{Y}{YM}\right)^{Y^- + M^-}
  \left(\sspre{P}{N}{YM}\right)^{N^+}
  \left(1 - \sspre{P}{N}{YM}\right)^{N^-}
  \stackrel{?}{>} \\
  \left(\sspre{P}{Y}{NM}\right)^{Y^+}
  \left(1 - \sspre{P}{Y}{NM}\right)^{Y^-}
  \left(\sspre{P}{N}{NM}\right)^{N^+ + M^+}
  \left(1 - \sspre{P}{N}{NM}\right)^{N^- + M^-}
\end{multline}


\section{Bayesian leaf probability estimation}

When the number of items in a leaf is few, it becomes difficult to estimate the
leaf probability with certainty. Probabilities of 0 and 1 are easy to achieve
with few data, and these do not scale well to new data. Any counter example at
all will drive the metric to zero. Hard to trust someone who claims perfection
then makes a mistake.

So, we assume a binomial distribution at the leaves, where $n$ is the number of
max bags there.

Two options considered so far: (1) find the expected value of $p$ given $k$ and
$n$, and (2) find the maximum a posteriori estimate given an beta prior. The
latter is easy to do and mainstream but requires a new parameter or two. Here is
my wishlist for a solution:

\begin{itemize}
\item Prevents extremes near 0 and 1 for low n.
\item Converges to k/n as n increases.
\item Doesn't change with meaningless additional expansions.
\item No new program parameters.
\item Fast.
\item Easy to implement.
\item Justifiable.
\end{itemize}

\end{document}
